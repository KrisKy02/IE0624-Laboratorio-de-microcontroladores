\section{Resumen}

En el presente laboratorio se desarrolló mediante el microcontrolador AT-tiny4313 un cruce de semáforos simplificado. El objetivo principal del laboratorio fue el estudio más a profundidad de temas como los GPIOs de un microcontrolador que se empezó a estudiar en el laboratorio 1, y la introducción de nuevos temas como fue el caso de interrupciones, timers y máquinas de estado en un microcontrolador.

Se utilizó el \textbf{software de simulación SimulIDE} para el diseño del circuito y se utilizaron elementos como luces LED, botones, capacitores y resistencias además del microcontrolador para poder construir existosamente el diseño. Además se utilizó el \textbf{lenguaje de programación C} para el desarrollo del código y poder programar las funcionalidades de los diferentes semáforos.

En resumen, el funcionamiento consiste en que se tienen dos semáforos peatonales y un semáforo vehicular. Al inicio de la simulación, se debe tomar que por al menos 10 segundos el semáforo vehicular debe estar en verde. Al pasar estos 10 segundos, se procede a presionar cualquiera de los dos botones de los semáforos peatonales indicando una interrupción y se empieza a correr la lógica implementada para el funcionamiento de los semáforos peatonales que será descrita con más profundidad más adelante en este reporte. 

Se puede encontrar el link del repositorio donde se encuentra contenido el proyecto en:
\textbf{\url{https://github.com/KrisKy02/Laboratorio2-microcontroladores}}